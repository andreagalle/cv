%%%%%%%%%%%%%%%%%
% This is an sample CV template created using altacv.cls
% (v1.7, 9 August 2023) written by LianTze Lim (liantze@gmail.com). Compiles with pdfLaTeX, XeLaTeX and LuaLaTeX.
%
%% It may be distributed and/or modified under the
%% conditions of the LaTeX Project Public License, either version 1.3
%% of this license or (at your option) any later version.
%% The latest version of this license is in
%%    http://www.latex-project.org/lppl.txt
%% and version 1.3 or later is part of all distributions of LaTeX
%% version 2003/12/01 or later.
%%%%%%%%%%%%%%%%

%% Use the "normalphoto" option if you want a normal photo instead of cropped to a circle
% \documentclass[10pt,a4paper,normalphoto]{altacv}

\documentclass[10pt,a4paper,ragged2e,withhyper]{altacv}
%% AltaCV uses the fontawesome5 and packages.
%% See http://texdoc.net/pkg/fontawesome5 for full list of symbols.

% Define a switch
\newif\iflongversion
% Set to true for the long version (single column), false for short version (two columns)
\longversiontrue  % Uncomment this line for long version
% \longversionfalse  % Uncomment this line for short version
\iflongversion
  % Single column layout for the long version
  \geometry{left=1.5cm,right=1.5cm,top=1.5cm,bottom=1.5cm}
\else
  % Two column layout for the short version
  \geometry{left=1.25cm,right=1.25cm,top=1.5cm,bottom=1.5cm,columnsep=1.2cm}
\fi

\usepackage{paracol}


% Change the font if you want to, depending on whether
% you're using pdflatex or xelatex/lualatex
% WHEN COMPILING WITH XELATEX PLEASE USE
% xelatex -shell-escape -output-driver="xdvipdfmx -z 0" sample.tex
\ifxetexorluatex
  % If using xelatex or lualatex:
  \setmainfont{Roboto Slab}
  \setsansfont{Lato}
  \renewcommand{\familydefault}{\sfdefault}
\else
  % If using pdflatex:
  \usepackage[rm]{roboto}
  \usepackage[defaultsans]{lato}
  % \usepackage{sourcesanspro}
  \renewcommand{\familydefault}{\sfdefault}
\fi

% Change the colours if you want to
\definecolor{SlateGrey}{HTML}{2E2E2E}
\definecolor{LightGrey}{HTML}{666666}
\definecolor{DarkPastelRed}{HTML}{450808}
\definecolor{PastelRed}{HTML}{8F0D0D}
\definecolor{GoldenEarth}{HTML}{E7D192}
\colorlet{name}{black}
\colorlet{tagline}{PastelRed}
\colorlet{heading}{DarkPastelRed}
\colorlet{headingrule}{GoldenEarth}
\colorlet{subheading}{PastelRed}
\colorlet{accent}{PastelRed}
\colorlet{emphasis}{SlateGrey}
\colorlet{body}{LightGrey}

% Change some fonts, if necessary
\renewcommand{\namefont}{\LARGE\rmfamily\bfseries}
\renewcommand{\personalinfofont}{\footnotesize}
\renewcommand{\cvsectionfont}{\large\rmfamily\bfseries}
\renewcommand{\cvsubsectionfont}{\large\bfseries}


% Change the bullets for itemize and rating marker
% for \cvskill if you want to
\renewcommand{\cvItemMarker}{{\small\textbullet}}
\renewcommand{\cvRatingMarker}{\faCircle}
% ...and the markers for the date/location for \cvevent
% \renewcommand{\cvDateMarker}{\faCalendar*[regular]}
% \renewcommand{\cvLocationMarker}{\faMapMarker*}


% If your CV/résumé is in a language other than English,
% then you probably want to change these so that when you
% copy-paste from the PDF or run pdftotext, the location
% and date marker icons for \cvevent will paste as correct
% translations. For example Spanish:
% \renewcommand{\locationname}{Ubicación}
% \renewcommand{\datename}{Fecha}


%% Use (and optionally edit if necessary) this .tex if you
%% want to use an author-year reference style like APA(6)
%% for your publication list
% % When using APA6 if you need more author names to be listed
% because you're e.g. the 12th author, add apamaxprtauth=12
\usepackage[backend=biber,style=apa6,sorting=ydnt]{biblatex}
\defbibheading{pubtype}{\cvsubsection{#1}}
\renewcommand{\bibsetup}{\vspace*{-\baselineskip}}
\AtEveryBibitem{%
  \makebox[\bibhang][l]{\itemmarker}%
  \iffieldundef{doi}{}{\clearfield{url}}%
}
\setlength{\bibitemsep}{0.25\baselineskip}
\setlength{\bibhang}{1.25em}


%% Use (and optionally edit if necessary) this .tex if you
%% want an originally numerical reference style like IEEE
%% for your publication list
\usepackage[backend=biber,style=ieee,sorting=ydnt,defernumbers=true]{biblatex}
%% For removing numbering entirely when using a numeric style
\setlength{\bibhang}{1.25em}
\DeclareFieldFormat{labelnumberwidth}{\makebox[\bibhang][l]{\itemmarker}}
\setlength{\biblabelsep}{0pt}
\defbibheading{pubtype}{\cvsubsection{#1}}
\renewcommand{\bibsetup}{\vspace*{-\baselineskip}}
\AtEveryBibitem{%
  \iffieldundef{doi}{}{\clearfield{url}}%
}


%% sample.bib contains your publications
\addbibresource{root.bib}

\begin{document}
\name{Andrea Gallegati, Ph.D}
\tagline{CSSLP – DevSecOps Engineer @ IDS AirNav -- ENAV Group}
%% You can add multiple photos on the left or right
\photoR{2.8cm}{bike}
% \photoL{2.5cm}{Yacht_High,Suitcase_High}

\personalinfo{%
	% Not all of these are required!
	\email{andrea.gallegati.job@gmail.com}
	\phone{+393381077321}
	% \mailaddress{Åddrésş, Street, 00000 Cóuntry}
	\location{Rome, IT} \\
	% \homepage{about.me/andreagalle}
	% \twitter{@GallegatiAndrea}
	\linkedin{/andrea-gallegati}
	\github{andreagalle}
	\orcid{0000-0001-7692-905X}
	%% You can add your own arbitrary detail with
	%% \printinfo{symbol}{detail}[optional hyperlink prefix]
	% \printinfo{\faPaw}{Hey ho!}[https://example.com/]

	%% Or you can declare your own field with
	%% \NewInfoFiled{fieldname}{symbol}[optional hyperlink prefix] and use it:
	% \NewInfoField{gitlab}{\faGitlab}[https://gitlab.com/]
	% \gitlab{your_id}
	%%
	%% For services and platforms like Mastodon where there isn't a
	%% straightforward relation between the user ID/nickname and the hyperlink,
	%% you can use \printinfo directly e.g.
	% \printinfo{\faMastodon}{@username@instace}[https://instance.url/@username]
	%% But if you absolutely want to create new dedicated info fields for
	%% such platforms, then use \NewInfoField* with a star:
	% \NewInfoField*{mastodon}{\faMastodon}
	%% then you can use \mastodon, with TWO arguments where the 2nd argument is
	%% the full hyperlink.
	% \mastodon{@username@instance}{https://instance.url/@username}
}

\makecvheader
%% Depending on your tastes, you may want to make fonts of itemize environments slightly smaller
% \AtBeginEnvironment{itemize}{\small}

\iflongversion
	% Single column version
	\cvsection{Experience}

\iflongversion
	\cvevent[IDS AirNav -- ENAV Group]{DevOps Engineer}{Jul 2024 -- Ongoing}{Rome, ITA}
	\begin{itemize}
		\item Managing several (backend) microservices, shared by any product in the suite, and some web apps, essential for the functionality and management of it.
		\item Maintaining and frequently updating third-party components to address rapidly emerging vulnerabilities.
		\item  Developing IaC and maintaining an installation and configuration tool based on Ansible, to standardize the deployment of solutions.
		\item Enabling a stable foundation that accelerates development by offering standardized tools, services, and integration patterns across the organization.
		\item Supporting operations and assisting customers when issues escalate beyond the third level of support.
	\end{itemize}
	\smallskip
	\cvtag{JHipster}\cvtag{Ansible}\cvtag{Postgres}\cvtag{Keycloak}\cvtag{DevSecOps}\cvtag{Docker Swarm}\cvtag{Postman}\cvtag{Jenkins}\cvtag{Burp Suite}

	\medskip

	\cvevent{Security Champion}{Mar 2023 -- Ongoing}{Rome, ITA}
	\begin{itemize}
		\item Certified Secure Software Lifecycle Professional (CSSLP – ISC2). Undergoing continuous training in cybersecurity.
		\item Assisting operations team in the deployment of containerized solutions in clustered environments, with Ansible.
		\item Leading threat modeling, dependency analysis, and security assessments (SAST/DAST findings, ASVS requirements), using advanced Jenkins pipelines to ensure a seamless CI/CD process.
	\end{itemize}
	\smallskip
	\cvtag{CSSLP}\cvtag{Threat Modeling}\cvtag{CIS}\cvtag{ASVS}\cvtag{OWASP}\cvtag{SCA}\cvtag{SAST}\cvtag{Defect Management}\cvtag{CI/CD}

	\medskip

	\cvevent{Software Engineer}{Apr 2021 -- Jul 2024}{Rome, ITA}
	\begin{itemize}
		\item Developing an high-load web application (for aeronautical publishing) using Spring Boot. A product sold by ENAV Group, the Italian Air Navigation Service Provider (ANSP) with hundreds of customers all around the world.
		\item Troubleshooting some performance issues, sometimes refactoring its microservice architecture, built around Apache Kafka, consuming massive data.
		\item Maintaining the legacy IAM solution (developed in-house) and taking part to the transition toward the current IAM solution (based on Keycloak).
		\item Designing and developing a migration tool for the transition from the legacy IAM solution. I made it in python, using Django for the migration tool dashboard.
		\item Organized two editions of a small hackathon at Sapienza University, using my network in the academy and challenging students with programming tasks. This brought new talent into our company.
	\end{itemize}
	\smallskip
	\cvtag{Java}\cvtag{Spring Boot}\cvtag{Apache Kafka}\cvtag{Django}\cvtag{Non-Functional Requirements}\cvtag{Microservices}\cvtag{Vaadin}

	\divider

	\cvevent[Temple University -- Rome Campus]{Intern Supervisor}{May 2024 -- Ongoing}{Rome, ITA}
	\begin{itemize}
		\item Sponsored student internships at Edukai, a startup where they developed and integrated services into the core product using advanced generative AI techniques and open-source projects in innovative ways.
		\item Supervised an intern working on the "Tiber River" project, developing a pipeline (in \emph{Python}) to automate the generation of 3D models of iconic monuments using OpenStreetMap data (OSM), advanced photogrammetry with generative AI, and automated 3D modeling refinements. %The pipeline was deployed in a cloud environment hosted by Huggingface.
	\end{itemize}
	\smallskip
	% \cvtag{Rapid Learning}\cvtag{Online Tutoring}\cvtag{Field Training}\cvtag{OpenStreetMap}\cvtag{Blender}\cvtag{Huggingface}

	\medskip

	\cvevent{Adjunct Instructor}{Jan 2019 -- Ongoing}{Rome, ITA}
	\begin{itemize}
		\item Teaching several classes, including \emph{Problem Solving \& Programming in Python}, \emph{Intermediate Algebra}, a foundational course that is part of the GenEd (General Education) curriculum and \emph{Mechanics of Solids}.
		\item Organized annually the \emph{Motor Valley} field trip to Modena, with workshops, seminars and meetings with motorsport experts (i.e. Dallara Academy, the Maserati factory and the Ferrari museum) for the \emph{Solids} class.
	\end{itemize}
	\smallskip
	\cvtag{Educational Program Design}\cvtag{Lesson Planning}\cvtag{University Teaching}\cvtag{Online Tutoring}
	\medskip

	\cvevent{Teaching Assistant}{Jan 2018 -- Dec 2019}{Rome, ITA}
	\begin{itemize}
		\item Teaching \emph{Mechanics of Solids} in English, to American mechanical engineering students.
		\item Gained significant insight into the Anglo-Saxon approach to education.
	\end{itemize}
	\smallskip
	% \cvtag{Assistant Teaching}\cvtag{English}

	\divider

	\cvevent[Cybertech.eu -- Engineering]{System Analyst Intern}{Apr 2021 -- Apr 2021}{Rome, ITA}
	\begin{itemize}
		\item Analyzing systems to implement appropriate Identity Access Management (IAM) solutions for a variety of clients in both Linux and Windows environments.
	\end{itemize}

	\divider

	\cvevent[Accenture]{Software Developer Intern}{Mar 2021 -- Mar 2021}{Rome, ITA}
	\begin{itemize}
		\item Developing a web-based management software for a department of the Italian Ministry of the Interior.
		\item Working on jBPM (Java Business Process Model) flows, to automate tasks and managing process instances.
	\end{itemize}

	\medskip

	\cvevent{Java Academy}{Dec 2020 -- Feb 2021}{Rome, ITA}
	\begin{itemize}
		\item Crash Course in Java (by Accenture), providing a significant leap into the world of software dev for technology.
	\end{itemize}

	\divider

	\cvevent[Sapienza University]{PhD Researcher}{Sep 2017 -- Jan 2021}{Rome, ITA}
	\begin{itemize}
		\item Conducting my research project, mentoring a few students each semester, that contributed to my own project.
	\end{itemize}
	\smallskip
	\cvtag{Research}\cvtag{Team Leading}\cvtag{Mentoring}\cvtag{Git}\cvtag{Agile}

	\medskip

	\cvevent{Teaching Assistant}{Mar 2019 -- Jun 2020}{Rome, ITA}
	\begin{itemize}
		\item Teaching assisting a computational aerodynamics lab, for the Department of Aerospace and Mechanical Engineering (DIMA), in a small class environment focused on hands-on learning.
		\item Training students on programming and applying basic aerodynamic principles to three-dimensional finite wings design. The code was developed in groups.
		\item Scouting students to recruit them, getting involved in departmental projects and join our research team.
	\end{itemize}
	\medskip

	\cvevent{Teaching Assistant}{Oct 2018 -- Dec 2019}{Rome, ITA}
	\begin{itemize}
		\item Teaching assisting a calculus class for the School of Industrial Engineering (ICI) to hundreds of freshman aerospace engineering students. Some of my former students are now working with me.
		\item Teaching assisting a Linear Algebra class for the Department of Basic Sciences Applied to Engineering (SBAI) to Biomedical, Energy, and Electrical Engineering classes.
		\item Interacting with students from different academic backgrounds and coordinating with different professors, with different teaching methods.
	\end{itemize}
\else
	\cvevent[IDS AirNav -- ENAV Group]{DevSecOps Engineer}{Apr 2021 -- Ongoing}{Rome, ITA}
	\begin{itemize}
		\item Appointed Security Champion, involved in DevSecOps \& automation to streamline Non-functional requirements.
		\item Developed a web-based migration tool in Python as part of the team maintaining the new IAM solution. Worked on several performance enhancements for the product suite.
	\end{itemize}
	\smallskip
	\cvtag{Java}\cvtag{Spring Boot}\cvtag{Kafka}\cvtag{python}\cvtag{Django}\cvtag{JS}\cvtag{bash} \\
	\cvtag{Docker}\cvtag{Jenkins}\cvtag{Ansible}\cvtag{OWASP}\cvtag{CIS}\cvtag{Keycloak}

	\divider

	\cvevent[Temple University -- Rome Campus]{Adjunct Instructor}{Jan 2018 -- Ongoing}{Rome, ITA}
	\begin{itemize}
		\item Teaching several classes, including Problem Solving \& Python.
	\end{itemize}
	% \smallskip
	\cvtag{Teaching}\cvtag{Planning}\cvtag{Public speaking}\cvtag{English}

	\divider

	\cvevent[Cybertech.eu -- Engineering]{System Analyst Intern}{Apr 2021 -- Apr 2021}{Rome, ITA}

	\divider

	\cvevent[Accenture Technology]{Software Developer Intern}{Dec 2020 -- Mar 2021}{Rome, ITA}

	\divider

	\cvevent[Sapienza University]{PhD Researcher \& Teaching Assistant}{Sep 2017 -- Jan 2021}{Rome, ITA}
	\begin{itemize}
		\item Conducting my research project, working with a few students each year, that contributed to my own project.
		\item TAing many classes, besides Aerodynamics Labs.
	\end{itemize}
	\smallskip
	\cvtag{Research}\cvtag{Team Leading}\cvtag{Mentoring}\cvtag{Git}\cvtag{Agile}

	\medskip

	\cvsection{A Day of My Life}

	% Adapted from @Jake's answer from http://tex.stackexchange.com/a/82729/226
	% \wheelchart{outer radius}{inner radius}{
	% comma-separated list of value/text width/color/detail}
	\wheelchart{1.5cm}{0.5cm}{%
		6/8em/accent!30/Sleep,
		3/8em/accent!40/Coding,
		8/8em/accent!60/My job,
		2/10em/accent/Teaching,
		5/6em/accent!20/My family
	}
	\switchcolumn
\fi

\iflongversion
	\cvsection{Projects}

	\cvevent[Temple University - Rome Campus]{walkintiber - Immersive Digital Twin of Rome}{Aug 2024 -- Present}{Rome, ITA}
	\begin{itemize}
		\item Co-founded a high-resolution, 3D digital twin of Rome accessible on the web, allowing users to virtually explore the Eternal City in immersive detail.
		\item Automated 24/7 pipeline to integrate live community data from OpenStreetMap (OSM), scaling the infrastructure horizontally for global coverage and vertically for enriched realism, including: textures, digital elevation models (DEM), and environmental detail.
		\item Integrating generative AI for texture creation and prototyped monument modeling using photogrammetry; developing VR compatibility for enhanced user engagement.
		\item Engaging TUR students and faculty through local "walks in Rome" contributing to OSM, and promoting hands-on learning in data mapping, AI, and VR technologies.
	\end{itemize}
	\smallskip
	This project demonstrates TUR's commitment to innovative educational experiences that blend advanced technology with cultural heritage. \\
	\smallskip
	\cvtag{IaC}\cvtag{Terraform}\cvtag{MongoDB Atlas}\cvtag{FastAPI}\cvtag{NestJS}\cvtag{Cloudflare}\cvtag{Apache Airflow}\cvtag{Blender}\cvtag{PlayCanvas}

	\divider

	\cvevent[IDS AirNav -- ENAV Group]{Migration Tool - Seamless transition from a legacy IAM solution}{Apr 2023 -- Oct 2023}{Rome, ITA}
	\begin{itemize}
		\item Designed the migration tool, consisting of two microservices: a Django web application for the dashboard, and Redis for real-time logging capabilities, leveraging Channels (within Django) and eliminating the need for polling.
		\item Automated the migration process to facilitate a seamless transition from a legacy IAM solution to a modern Keycloak-based IAM system, ensuring minimal downtime and business continuity for users.
		\item Provided two deployment options: a full-featured version with a graphical interface for configuration and tracking, and a lightweight "headless" version allowing CLI-driven migrations.
		\item Integrated a bunch of different clients (HTTP, LDAP, PostgreSQL, and SSH using \emph{paramiko}) for accessing and retrieving configurations from the legacy system, ensuring validation features on the dashboard itself.
	\end{itemize}
	\smallskip
	This not only provided a structured approach to the transition, but also enabled detailed tracking and verification, ensuring data integrity and system resilience.

	\divider

	\cvevent[NASA SpaceApps Challenge]{CROPP - A whole new way of farming}{Apr 2015 - Oct 2015}{Rome, ITA}
	\begin{itemize}
		\item First-prize winning project. Easy-to-use platform, for farmers, to monitor and address crop-related threats.
		\item On-field sensors embedded in distributed hardware devices. Imagery from Low Earth Orbit (LEO) satellites.
		\item A 48-hour hackathon, to develop a working prototype for the IoT device and an Android mobile app.
		\item Demo at the World Expo hosted by Milan, in Italy, themed "Feeding the Planet, Energy for Life."
		\item Pitched the project to NASA’s then-Administrator, Charles Bolden Jr. at the U.S. Pavilion.
	\end{itemize}

	\cvsection{Talks}
	\cvevent[IDS AirNav -- ENAV Group]{AeroSIG 20th Edition}{11 -- 12 Oct 2022}{Rome, ITA}
	The Aeronautical Special Interest Group (AeroSIG) conference gathered colleagues and industry experts from over 35 countries, featuring a plenary session on the evolution of IDS AirNav solutions and breakout sessions focused on advanced topics in Air Traffic Management (ATM) and Aeronautical Information Management (AIM). \\
	\medskip
	Presentation title: ``IAM Status and Updates.'' \\
	\smallskip
	\begin{itemize}
		\item Outlining the transition from the legacy IAM solution to a new system, based on Keycloak and customizations.
		\item Addressing IAM modernization strategies, detailing how to leverage Keycloak for improved security/interoperability.
	\end{itemize}

	\divider

	\cvevent[AIMETA - Ass. Italiana Meccanica Teorica e Applicata]{XXIV Congress AIMETA 2019}{15 - 19 Sep 2019}{Rome, ITA}
	The AIMETA Congress is a prominent event fostering cross-disciplinary collaboration in mechanics, mathematics, and applied sciences. The event is recognized for bridging Rational Mechanics and Mathematical Physics with applications in fields such as biomechanics and material science. \\
	\medskip
	Presentation title: ``Droplet Homogeneous Nucleation in Two-Way Coupled Turbulent Flows''

	\divider

	\cvevent[EUROMECH Conference]{ETC17 - European Turbulence Conference}{3 - 6 Sep 2019}{Turin, ITA}
	The European Turbulence Conference (ETC) is a bi-annual event providing a platform for over 450 scientists to present recent advances in turbulence research. ETC17 gathered experts in flow dynamics, offering insights into cutting-edge developments in the field. \\
	\medskip
	Presentation title: ``Droplet Nucleation in Turbulent Steam Jets''

	\divider

	\cvevent[ERCOFTAC Workshop]{DLES12 - Direct and Large Eddy Simulation Workshop}{5 - 7 Jun 2019}{Madrid, ESP}
	The ERCOFTAC DLES Workshop is dedicated to advancements in turbulent flow simulation, focusing on Direct Numerical Simulation (DNS) and Large Eddy Simulation (LES) techniques. The 2019 edition gathered specialists to discuss advances in computational modeling of transitional and turbulent flows. \\
	\medskip
	Presentation title: ``Droplet Homogeneous Nucleation in a Turbulent Steam Jet in the Two-Way Coupling Regime''

	\bigskip

	% % use ONLY \newpage if you want to force a page break for
	% % ONLY the current column
	% \newpage

	\cvsection{Publications}

	%% Specify your last name(s) and first name(s) as given in the .bib to automatically bold your own name in the publications list.
	%% One caveat: You need to write \bibnamedelima where there's a space in your name for this to work properly; or write \bibnamedelimi if you use initials in the .bib
	%% You can specify multiple names, especially if you have changed your name or if you need to highlight multiple authors.
	\mynames{Gallegati/A.}
	%% MAKE SURE THERE IS NO SPACE AFTER THE FINAL NAME IN YOUR \mynames LIST

	\nocite{*}

	% \printbibliography[heading=pubtype,title={\printinfo{\faBook}{Books}},type=book]

	% \divider

	% \printbibliography[heading=pubtype,title={\printinfo{\faFile*[regular]}{Journal Articles}},type=article]

	% \divider

	\printbibliography[heading=pubtype,title={\printinfo{\faUsers}{Conference Proceedings}},type=inproceedings]

	%% Switch to the right column. This will now automatically move to the second
	%% page if the content is too long.

	\bigskip

	\columnratio{0.5}
	% Start a 2-column paracol. Both the left and right columns will automatically break across pages if things get too long.
	\begin{paracol}{2}
		% Two-column content goes here

		\cvsection{Awards}

		\cvevent[Fondazione Roma Sapienza]{Excellent graduate student - 6th edition}{16 Apr 2018}{Rome, ITA}
		Awarded among the best graduates of 2017 at Sapienza.

		\bigskip

		\cvevent[NASA - Kennedy Space Center (KSC)]{Galactic Impact Prize}{2 Sep 2015}{Cape Canaveral, USA}
		Awarded and invited to attend a scheduled rocket launch.

		\switchcolumn

		\cvsection{Grants}

		\cvevent[Temple University]{Professional Development Grant}{Oct 2024 - Jul 2025}{Rome, ITA}
		Funding to support research and creative work aimed at promoting interdisciplinary learning and global engagement. This involves the application of advanced technologies such as generative AI and VR.

		\bigskip

		\cvevent[CINECA]{ISCRA Class C Projects - HP10CEKU6J}{May 2020 - Feb 2021}{Rome, ITA}
		\cvevent{ISCRA Class C Projects - HP10CQBSIR}{Oct 2018 - Nov 2019}{Rome, ITA}
		Resources on HPC facilities for advanced numerical simulations in fluid dynamics, leveraging up to 100,000 core hours for heavy workloads and visualization tasks.
		% Collaborated with CINECA’s User Support Group to optimize the application’s performance, ensuring efficient utilization of high-performance computing resources. \\


		% Secured computational resources for a test and development project utilizing CINECA’s HPC systems to explore cutting-edge methods in turbulent flow simulation. Received dedicated specialist support to enable and optimize applications, allowing for breakthroughs in modeling techniques and enhanced computational efficiency in the field of fluid dynamics.

	\end{paracol}
\else
	\cvsection{Projects}

	\cvevent[Temple University - Rome Campus]{walkintiber - 3D Digital Twin of Rome}{Aug 2024 -- Present}{Rome, ITA}
	\begin{itemize}
		\item Co-founded an immersive 3D model of Rome, allowing users to explore the city virtually via a web interface.
		\item Built a pipeline integrating live data from OpenStreetMap (OSM) to continuously update and enhance the model with realistic textures and digital elevation, supported by generative AI.
		\item Engaged TUR students and faculty in "walks in Rome" for hands-on learning in data mapping and VR technologies.
	\end{itemize}
	\smallskip
\fi

\bigskip

\cvsection{Education}

\iflongversion

	\cvevent[Sapienza University -- DIMA]{Ph.D.\ in Theoretical \& Applied Mechanics}{Nov 2017 -- May 2021}{Rome, ITA}
	Thesis title: ``Droplet condensation in turbulent jets.'' Supervised by: Prof. Paolo Gualtieri. \\
	\smallskip
	\begin{itemize}
		\item Developing high-performance software (HPC) for simulating complex, high-resolution, multi-phase turbulent flows.
		\item Maintaining optimized codebases, using parallel computing paradigms (OpenMP, MPI, CUDA) to take full advantage of distributed computing facilities
		\item Deriving an analytical approach to accurately represent phase transition processes interaction, in turbulent flows.
		\item Building a pipeline to deploy the compiled code onto remote supercomputers, running simulations on different environments and to periodically post-process intermediate results, monitoring the status real-time.
	\end{itemize}
	\medskip
	\cvtag{Turbulence}\cvtag{Multiphase Flows}\cvtag{HPC}\cvtag{Direct Numerical Simulations}\cvtag{Fortran}\
	\medskip

	\cvevent{M.Sc.\ in Aeronautical Engineering}{Sept 2014 -- July 2017}{Rome, ITA}
	\textbf{110/110 cum laude} Thesis title: ``A consistent framework for mass, momentum and energy exchange in two phase flows: DNS of a turbulent jet in the two-way coupling regime.'' Supervised by: Prof Carlo Massimo Casciola. \\
	\smallskip
	\begin{itemize}
		\item Specialized into aerodynamics, propulsion, and structures (APS) curriculum.
		\item Focused on continuum mechanics, applying fundamental science to solve complex engineering problems.
		\item Fascinated by theoretical approach and numerical methods.
	\end{itemize}
	\medskip
	\cvtag{Fortran}\cvtag{PDE}\cvtag{Navier-Stokes Eqs.}

	\divider

	\cvevent[ISAE Supaero -- DAEP]{Erasmus+ Program}{Jan 2016 -- Aug 2016}{Toulouse, FRA}
	Project title: ``Cost versus accuracy in LES of wall bounded flows in CharlesX.'' Supervised by: Prof. Julien Bodart. \\
	\medskip
	\begin{itemize}
		\item Studied in an international environment, in the heart of an aerospace hub in Europe, with many different cultures.
		\item Experienced with enthusiasms the multidisciplinary  approach, that is typical of France and that has enriched me so much from a technical point of view.
		\item Developed Python code for post-processing fluid dynamics numerical simulations, benchmarking a software developed by the Center for Turbulence Research (CTR), in Stanford.
	\end{itemize}
	\medskip
	\cvtag{python}\cvtag{C++}\cvtag{bash}\cvtag{French}

	\divider

	\cvevent[Sapienza University]{B.Sc.\ in Aerospace Engineering}{Sept 2011 -- Nov 2014}{Rome, ITA}
	\textbf{110/110 cum laude} Thesis title: ``The Newton's problem of minimal resistance.'' Supervised by: Prof. Andrea Dall'Aglio. \\
	\medskip
	% Aerospace Engineering is where the convergence of hard science and diverse technologies takes place to advance humanity's outer reach, with a focus on integrating diverse technological solutions in a collaborative, often international environment. This dual approach reflects my belief that true innovation often comes from a deep understanding of fundamental principles and culminated in my BSc thesis.
	% \medskip
	\cvtag{Functional Analysis}\cvtag{\LaTeX}

	\divider

	\cvevent[LSS J.F. Kennedy]{High School Diploma}{Sept 2006 -- July 2011}{Rome, ITA}
	\textbf{87/100} Baccalaureate thesis: ``Revolutions in history and scientific thought.''
	\medskip\
	\begin{itemize}
		\item Playing competitive water polo with daily training. I learned a lot about cooperation, discipline and resilience, both in and out of the pool.
		\item Advanced education program (PNI), including intensive study of scientific subjects such as mathematics and physics, complemented by the use of computer science as a tool.
	\end{itemize}

	\bigskip

	\columnratio{0.55}
	% Start a 2-column paracol. Both the left and right columns will automatically break across pages if things get too long.
	\begin{paracol}{2}
		% Two-column content goes here

		\cvsection{Referees}

		% \cvref{name}{email}{mailing address}
		\cvref{Prof.\ Andrea Dall'Aglio}{Sapienza University}{dallaglio@mat.uniroma1.it}
		%{Address Line 1\\Address line 2}
		\cvref{Prof.\ Mary Conran}{Temple University Rome Campus}{mary.conran@temple.edu}
		%{Address Line 1\\Address line 2}
		\cvref{Prof.\ Carlo Massimo Casciola}{Sapienza University}{carlomassimo.casciola@uniroma1.it}
		\cvref{Prof.\ Paolo Gualtieri}{Sapienza University}{paolo.gualtieri@uniroma1.it}
		\cvref{Leonardo Moavero}{IDS AirNav}{leonardo.moavero@idsairnav.com}
		\cvref{Valerio Ferrara}{ENAV Group}{valerio.ferrara@enav.it}
		\cvref{Prof.\ Julien Bodart}{ISAE Supaero}{julien.bodart@isae.fr}

		\switchcolumn

		\cvsection{Languages}

		\cvskill{English (Full Professional)}{4.5}
		\cvskill{French}{3.5} %% Supports X.5 values.
		\cvskill{Italian (Native)}{5}

		% \cvsection{Strengths}

		% \cvtag{Hard-working}
		% \cvtag{Eye for detail}\\
		% \cvtag{Motivator \& Leader}

		% \divider\smallskip

		% \cvtag{C++}
		% \cvtag{Embedded Systems}\\
		% \cvtag{Statistical Analysis}

		\bigskip

		\cvsection{More}

		I love working in diverse teams, fostering growth, and building strong relationships with my colleagues. \par

		\smallskip

		Outside of work, you'll find me cooking with friends and family, discussing movies of all genres, staying active, or diving into late night coding sessions. \par

		\smallskip

		I'm always looking for opportunities to grow, share knowledge, and make a positive impact. I free-dive into the blue and look around. I like to take action.

	\end{paracol}

\else

	\cvevent[Sapienza University -- DIMA]{Ph.D.\ in Theoretical \& Applied Mechanics}{Nov 2017 -- May 2021}{Rome, ITA}
	``Droplet condensation in turbulent jets.'' \\
	\medskip
	\cvtag{HPC}\cvtag{MPI}\cvtag{C}\cvtag{Multiphase Flows}

	\divider

	\cvevent[Sapienza University -- DIMA]{M.Sc.\ in Aeronautical Engineering}{Sept 2014 -- July 2017}{Rome, ITA}
	``A consistent framework for mass, momentum and energy exchange in two phase flows: DNS of a turbulent jet in the two-way coupling regime.'' \\
	\medskip
	110/110 cum laude \\
	\medskip
	\cvtag{Fortran}\cvtag{PDE}\cvtag{Navier-Stokes Eqs.}

	\divider

	\cvevent[ISAE Supaero -- DAEP]{Erasmus+ Program}{Jan 2016 -- Aug 2016}{Toulouse, FRA}
	``Cost versus accuracy in LES of wall bounded flows in CharlesX.'' \\
	\medskip
	\cvtag{python}\cvtag{C++}\cvtag{bash}\cvtag{French}

	\divider

	% \cvevent[Sapienza University]{B.Sc.\ in Aerospace Engineering}{Sept 2011 -- Nov 2014}{Rome, ITA}
	% ``The Newton's problem of minimal resistance.'' \\
	% \medskip
	% 110/110 cum laude \\
	% \medskip
	% \cvtag{Functional Analysis}\cvtag{\LaTeX}

	% \divider

	% \cvevent[LSS J.F. Kennedy]{High School Diploma}{Sept 2006 -- July 2011}{Rome, ITA}
	% 87/100

	\bigskip

	\cvsection{Languages}

	\cvskill{English (Full Professional)}{4.5}
	\cvskill{French}{3.5} %% Supports X.5 values.
	\cvskill{Italian (Native)}{5}

\fi

\divider

Last updated: \today

% \cvsection{My Life Philosophy}

% \begin{quote}
% 	``Whatever Works.'' comedy film directed and written by Woody Allen
% \end{quote}

% \cvsection{Most Proud of}

% \cvachievement{\faTrophy}{Fantastic Achievement}{and some details about it}

% \divider

% \cvachievement{\faHeartbeat}{Another achievement}{more details about it of course}

% \divider

% \cvachievement{\faHeartbeat}{Another achievement}{more details about it of course}

% %% Yeah I didn't spend too much time making all the
% %% spacing consistent... sorry. Use \smallskip, \medskip,
% %% \bigskip, \vspace etc to make adjustments.
% \medskip

\else
	% The paracol package lets you typeset columns of text in parallel
	% Set the left/right column width ratio to 6:4.
	\columnratio{0.55}
	% Start a 2-column paracol. Both the left and right columns will automatically
	% break across pages if things get too long.
	\begin{paracol}{2}
		% Two-column content goes here
		\cvsection{Experience}

\iflongversion
	\cvevent[IDS AirNav -- ENAV Group]{DevOps Engineer}{Jul 2024 -- Ongoing}{Rome, ITA}
	\begin{itemize}
		\item Managing several (backend) microservices, shared by any product in the suite, and some web apps, essential for the functionality and management of it.
		\item Maintaining and frequently updating third-party components to address rapidly emerging vulnerabilities.
		\item  Developing IaC and maintaining an installation and configuration tool based on Ansible, to standardize the deployment of solutions.
		\item Enabling a stable foundation that accelerates development by offering standardized tools, services, and integration patterns across the organization.
		\item Supporting operations and assisting customers when issues escalate beyond the third level of support.
	\end{itemize}
	\smallskip
	\cvtag{JHipster}\cvtag{Ansible}\cvtag{Postgres}\cvtag{Keycloak}\cvtag{DevSecOps}\cvtag{Docker Swarm}\cvtag{Postman}\cvtag{Jenkins}\cvtag{Burp Suite}

	\medskip

	\cvevent{Security Champion}{Mar 2023 -- Ongoing}{Rome, ITA}
	\begin{itemize}
		\item Certified Secure Software Lifecycle Professional (CSSLP – ISC2). Undergoing continuous training in cybersecurity.
		\item Assisting operations team in the deployment of containerized solutions in clustered environments, with Ansible.
		\item Leading threat modeling, dependency analysis, and security assessments (SAST/DAST findings, ASVS requirements), using advanced Jenkins pipelines to ensure a seamless CI/CD process.
	\end{itemize}
	\smallskip
	\cvtag{CSSLP}\cvtag{Threat Modeling}\cvtag{CIS}\cvtag{ASVS}\cvtag{OWASP}\cvtag{SCA}\cvtag{SAST}\cvtag{Defect Management}\cvtag{CI/CD}

	\medskip

	\cvevent{Software Engineer}{Apr 2021 -- Jul 2024}{Rome, ITA}
	\begin{itemize}
		\item Developing an high-load web application (for aeronautical publishing) using Spring Boot. A product sold by ENAV Group, the Italian Air Navigation Service Provider (ANSP) with hundreds of customers all around the world.
		\item Troubleshooting some performance issues, sometimes refactoring its microservice architecture, built around Apache Kafka, consuming massive data.
		\item Maintaining the legacy IAM solution (developed in-house) and taking part to the transition toward the current IAM solution (based on Keycloak).
		\item Designing and developing a migration tool for the transition from the legacy IAM solution. I made it in python, using Django for the migration tool dashboard.
		\item Organized two editions of a small hackathon at Sapienza University, using my network in the academy and challenging students with programming tasks. This brought new talent into our company.
	\end{itemize}
	\smallskip
	\cvtag{Java}\cvtag{Spring Boot}\cvtag{Apache Kafka}\cvtag{Django}\cvtag{Non-Functional Requirements}\cvtag{Microservices}\cvtag{Vaadin}

	\divider

	\cvevent[Temple University -- Rome Campus]{Intern Supervisor}{May 2024 -- Ongoing}{Rome, ITA}
	\begin{itemize}
		\item Sponsored student internships at Edukai, a startup where they developed and integrated services into the core product using advanced generative AI techniques and open-source projects in innovative ways.
		\item Supervised an intern working on the "Tiber River" project, developing a pipeline (in \emph{Python}) to automate the generation of 3D models of iconic monuments using OpenStreetMap data (OSM), advanced photogrammetry with generative AI, and automated 3D modeling refinements. %The pipeline was deployed in a cloud environment hosted by Huggingface.
	\end{itemize}
	\smallskip
	% \cvtag{Rapid Learning}\cvtag{Online Tutoring}\cvtag{Field Training}\cvtag{OpenStreetMap}\cvtag{Blender}\cvtag{Huggingface}

	\medskip

	\cvevent{Adjunct Instructor}{Jan 2019 -- Ongoing}{Rome, ITA}
	\begin{itemize}
		\item Teaching several classes, including \emph{Problem Solving \& Programming in Python}, \emph{Intermediate Algebra}, a foundational course that is part of the GenEd (General Education) curriculum and \emph{Mechanics of Solids}.
		\item Organized annually the \emph{Motor Valley} field trip to Modena, with workshops, seminars and meetings with motorsport experts (i.e. Dallara Academy, the Maserati factory and the Ferrari museum) for the \emph{Solids} class.
	\end{itemize}
	\smallskip
	\cvtag{Educational Program Design}\cvtag{Lesson Planning}\cvtag{University Teaching}\cvtag{Online Tutoring}
	\medskip

	\cvevent{Teaching Assistant}{Jan 2018 -- Dec 2019}{Rome, ITA}
	\begin{itemize}
		\item Teaching \emph{Mechanics of Solids} in English, to American mechanical engineering students.
		\item Gained significant insight into the Anglo-Saxon approach to education.
	\end{itemize}
	\smallskip
	% \cvtag{Assistant Teaching}\cvtag{English}

	\divider

	\cvevent[Cybertech.eu -- Engineering]{System Analyst Intern}{Apr 2021 -- Apr 2021}{Rome, ITA}
	\begin{itemize}
		\item Analyzing systems to implement appropriate Identity Access Management (IAM) solutions for a variety of clients in both Linux and Windows environments.
	\end{itemize}

	\divider

	\cvevent[Accenture]{Software Developer Intern}{Mar 2021 -- Mar 2021}{Rome, ITA}
	\begin{itemize}
		\item Developing a web-based management software for a department of the Italian Ministry of the Interior.
		\item Working on jBPM (Java Business Process Model) flows, to automate tasks and managing process instances.
	\end{itemize}

	\medskip

	\cvevent{Java Academy}{Dec 2020 -- Feb 2021}{Rome, ITA}
	\begin{itemize}
		\item Crash Course in Java (by Accenture), providing a significant leap into the world of software dev for technology.
	\end{itemize}

	\divider

	\cvevent[Sapienza University]{PhD Researcher}{Sep 2017 -- Jan 2021}{Rome, ITA}
	\begin{itemize}
		\item Conducting my research project, mentoring a few students each semester, that contributed to my own project.
	\end{itemize}
	\smallskip
	\cvtag{Research}\cvtag{Team Leading}\cvtag{Mentoring}\cvtag{Git}\cvtag{Agile}

	\medskip

	\cvevent{Teaching Assistant}{Mar 2019 -- Jun 2020}{Rome, ITA}
	\begin{itemize}
		\item Teaching assisting a computational aerodynamics lab, for the Department of Aerospace and Mechanical Engineering (DIMA), in a small class environment focused on hands-on learning.
		\item Training students on programming and applying basic aerodynamic principles to three-dimensional finite wings design. The code was developed in groups.
		\item Scouting students to recruit them, getting involved in departmental projects and join our research team.
	\end{itemize}
	\medskip

	\cvevent{Teaching Assistant}{Oct 2018 -- Dec 2019}{Rome, ITA}
	\begin{itemize}
		\item Teaching assisting a calculus class for the School of Industrial Engineering (ICI) to hundreds of freshman aerospace engineering students. Some of my former students are now working with me.
		\item Teaching assisting a Linear Algebra class for the Department of Basic Sciences Applied to Engineering (SBAI) to Biomedical, Energy, and Electrical Engineering classes.
		\item Interacting with students from different academic backgrounds and coordinating with different professors, with different teaching methods.
	\end{itemize}
\else
	\cvevent[IDS AirNav -- ENAV Group]{DevSecOps Engineer}{Apr 2021 -- Ongoing}{Rome, ITA}
	\begin{itemize}
		\item Appointed Security Champion, involved in DevSecOps \& automation to streamline Non-functional requirements.
		\item Developed a web-based migration tool in Python as part of the team maintaining the new IAM solution. Worked on several performance enhancements for the product suite.
	\end{itemize}
	\smallskip
	\cvtag{Java}\cvtag{Spring Boot}\cvtag{Kafka}\cvtag{python}\cvtag{Django}\cvtag{JS}\cvtag{bash} \\
	\cvtag{Docker}\cvtag{Jenkins}\cvtag{Ansible}\cvtag{OWASP}\cvtag{CIS}\cvtag{Keycloak}

	\divider

	\cvevent[Temple University -- Rome Campus]{Adjunct Instructor}{Jan 2018 -- Ongoing}{Rome, ITA}
	\begin{itemize}
		\item Teaching several classes, including Problem Solving \& Python.
	\end{itemize}
	% \smallskip
	\cvtag{Teaching}\cvtag{Planning}\cvtag{Public speaking}\cvtag{English}

	\divider

	\cvevent[Cybertech.eu -- Engineering]{System Analyst Intern}{Apr 2021 -- Apr 2021}{Rome, ITA}

	\divider

	\cvevent[Accenture Technology]{Software Developer Intern}{Dec 2020 -- Mar 2021}{Rome, ITA}

	\divider

	\cvevent[Sapienza University]{PhD Researcher \& Teaching Assistant}{Sep 2017 -- Jan 2021}{Rome, ITA}
	\begin{itemize}
		\item Conducting my research project, working with a few students each year, that contributed to my own project.
		\item TAing many classes, besides Aerodynamics Labs.
	\end{itemize}
	\smallskip
	\cvtag{Research}\cvtag{Team Leading}\cvtag{Mentoring}\cvtag{Git}\cvtag{Agile}

	\medskip

	\cvsection{A Day of My Life}

	% Adapted from @Jake's answer from http://tex.stackexchange.com/a/82729/226
	% \wheelchart{outer radius}{inner radius}{
	% comma-separated list of value/text width/color/detail}
	\wheelchart{1.5cm}{0.5cm}{%
		6/8em/accent!30/Sleep,
		3/8em/accent!40/Coding,
		8/8em/accent!60/My job,
		2/10em/accent/Teaching,
		5/6em/accent!20/My family
	}
	\switchcolumn
\fi

\iflongversion
	\cvsection{Projects}

	\cvevent[Temple University - Rome Campus]{walkintiber - Immersive Digital Twin of Rome}{Aug 2024 -- Present}{Rome, ITA}
	\begin{itemize}
		\item Co-founded a high-resolution, 3D digital twin of Rome accessible on the web, allowing users to virtually explore the Eternal City in immersive detail.
		\item Automated 24/7 pipeline to integrate live community data from OpenStreetMap (OSM), scaling the infrastructure horizontally for global coverage and vertically for enriched realism, including: textures, digital elevation models (DEM), and environmental detail.
		\item Integrating generative AI for texture creation and prototyped monument modeling using photogrammetry; developing VR compatibility for enhanced user engagement.
		\item Engaging TUR students and faculty through local "walks in Rome" contributing to OSM, and promoting hands-on learning in data mapping, AI, and VR technologies.
	\end{itemize}
	\smallskip
	This project demonstrates TUR's commitment to innovative educational experiences that blend advanced technology with cultural heritage. \\
	\smallskip
	\cvtag{IaC}\cvtag{Terraform}\cvtag{MongoDB Atlas}\cvtag{FastAPI}\cvtag{NestJS}\cvtag{Cloudflare}\cvtag{Apache Airflow}\cvtag{Blender}\cvtag{PlayCanvas}

	\divider

	\cvevent[IDS AirNav -- ENAV Group]{Migration Tool - Seamless transition from a legacy IAM solution}{Apr 2023 -- Oct 2023}{Rome, ITA}
	\begin{itemize}
		\item Designed the migration tool, consisting of two microservices: a Django web application for the dashboard, and Redis for real-time logging capabilities, leveraging Channels (within Django) and eliminating the need for polling.
		\item Automated the migration process to facilitate a seamless transition from a legacy IAM solution to a modern Keycloak-based IAM system, ensuring minimal downtime and business continuity for users.
		\item Provided two deployment options: a full-featured version with a graphical interface for configuration and tracking, and a lightweight "headless" version allowing CLI-driven migrations.
		\item Integrated a bunch of different clients (HTTP, LDAP, PostgreSQL, and SSH using \emph{paramiko}) for accessing and retrieving configurations from the legacy system, ensuring validation features on the dashboard itself.
	\end{itemize}
	\smallskip
	This not only provided a structured approach to the transition, but also enabled detailed tracking and verification, ensuring data integrity and system resilience.

	\divider

	\cvevent[NASA SpaceApps Challenge]{CROPP - A whole new way of farming}{Apr 2015 - Oct 2015}{Rome, ITA}
	\begin{itemize}
		\item First-prize winning project. Easy-to-use platform, for farmers, to monitor and address crop-related threats.
		\item On-field sensors embedded in distributed hardware devices. Imagery from Low Earth Orbit (LEO) satellites.
		\item A 48-hour hackathon, to develop a working prototype for the IoT device and an Android mobile app.
		\item Demo at the World Expo hosted by Milan, in Italy, themed "Feeding the Planet, Energy for Life."
		\item Pitched the project to NASA’s then-Administrator, Charles Bolden Jr. at the U.S. Pavilion.
	\end{itemize}

	\cvsection{Talks}
	\cvevent[IDS AirNav -- ENAV Group]{AeroSIG 20th Edition}{11 -- 12 Oct 2022}{Rome, ITA}
	The Aeronautical Special Interest Group (AeroSIG) conference gathered colleagues and industry experts from over 35 countries, featuring a plenary session on the evolution of IDS AirNav solutions and breakout sessions focused on advanced topics in Air Traffic Management (ATM) and Aeronautical Information Management (AIM). \\
	\medskip
	Presentation title: ``IAM Status and Updates.'' \\
	\smallskip
	\begin{itemize}
		\item Outlining the transition from the legacy IAM solution to a new system, based on Keycloak and customizations.
		\item Addressing IAM modernization strategies, detailing how to leverage Keycloak for improved security/interoperability.
	\end{itemize}

	\divider

	\cvevent[AIMETA - Ass. Italiana Meccanica Teorica e Applicata]{XXIV Congress AIMETA 2019}{15 - 19 Sep 2019}{Rome, ITA}
	The AIMETA Congress is a prominent event fostering cross-disciplinary collaboration in mechanics, mathematics, and applied sciences. The event is recognized for bridging Rational Mechanics and Mathematical Physics with applications in fields such as biomechanics and material science. \\
	\medskip
	Presentation title: ``Droplet Homogeneous Nucleation in Two-Way Coupled Turbulent Flows''

	\divider

	\cvevent[EUROMECH Conference]{ETC17 - European Turbulence Conference}{3 - 6 Sep 2019}{Turin, ITA}
	The European Turbulence Conference (ETC) is a bi-annual event providing a platform for over 450 scientists to present recent advances in turbulence research. ETC17 gathered experts in flow dynamics, offering insights into cutting-edge developments in the field. \\
	\medskip
	Presentation title: ``Droplet Nucleation in Turbulent Steam Jets''

	\divider

	\cvevent[ERCOFTAC Workshop]{DLES12 - Direct and Large Eddy Simulation Workshop}{5 - 7 Jun 2019}{Madrid, ESP}
	The ERCOFTAC DLES Workshop is dedicated to advancements in turbulent flow simulation, focusing on Direct Numerical Simulation (DNS) and Large Eddy Simulation (LES) techniques. The 2019 edition gathered specialists to discuss advances in computational modeling of transitional and turbulent flows. \\
	\medskip
	Presentation title: ``Droplet Homogeneous Nucleation in a Turbulent Steam Jet in the Two-Way Coupling Regime''

	\bigskip

	% % use ONLY \newpage if you want to force a page break for
	% % ONLY the current column
	% \newpage

	\cvsection{Publications}

	%% Specify your last name(s) and first name(s) as given in the .bib to automatically bold your own name in the publications list.
	%% One caveat: You need to write \bibnamedelima where there's a space in your name for this to work properly; or write \bibnamedelimi if you use initials in the .bib
	%% You can specify multiple names, especially if you have changed your name or if you need to highlight multiple authors.
	\mynames{Gallegati/A.}
	%% MAKE SURE THERE IS NO SPACE AFTER THE FINAL NAME IN YOUR \mynames LIST

	\nocite{*}

	% \printbibliography[heading=pubtype,title={\printinfo{\faBook}{Books}},type=book]

	% \divider

	% \printbibliography[heading=pubtype,title={\printinfo{\faFile*[regular]}{Journal Articles}},type=article]

	% \divider

	\printbibliography[heading=pubtype,title={\printinfo{\faUsers}{Conference Proceedings}},type=inproceedings]

	%% Switch to the right column. This will now automatically move to the second
	%% page if the content is too long.

	\bigskip

	\columnratio{0.5}
	% Start a 2-column paracol. Both the left and right columns will automatically break across pages if things get too long.
	\begin{paracol}{2}
		% Two-column content goes here

		\cvsection{Awards}

		\cvevent[Fondazione Roma Sapienza]{Excellent graduate student - 6th edition}{16 Apr 2018}{Rome, ITA}
		Awarded among the best graduates of 2017 at Sapienza.

		\bigskip

		\cvevent[NASA - Kennedy Space Center (KSC)]{Galactic Impact Prize}{2 Sep 2015}{Cape Canaveral, USA}
		Awarded and invited to attend a scheduled rocket launch.

		\switchcolumn

		\cvsection{Grants}

		\cvevent[Temple University]{Professional Development Grant}{Oct 2024 - Jul 2025}{Rome, ITA}
		Funding to support research and creative work aimed at promoting interdisciplinary learning and global engagement. This involves the application of advanced technologies such as generative AI and VR.

		\bigskip

		\cvevent[CINECA]{ISCRA Class C Projects - HP10CEKU6J}{May 2020 - Feb 2021}{Rome, ITA}
		\cvevent{ISCRA Class C Projects - HP10CQBSIR}{Oct 2018 - Nov 2019}{Rome, ITA}
		Resources on HPC facilities for advanced numerical simulations in fluid dynamics, leveraging up to 100,000 core hours for heavy workloads and visualization tasks.
		% Collaborated with CINECA’s User Support Group to optimize the application’s performance, ensuring efficient utilization of high-performance computing resources. \\


		% Secured computational resources for a test and development project utilizing CINECA’s HPC systems to explore cutting-edge methods in turbulent flow simulation. Received dedicated specialist support to enable and optimize applications, allowing for breakthroughs in modeling techniques and enhanced computational efficiency in the field of fluid dynamics.

	\end{paracol}
\else
	\cvsection{Projects}

	\cvevent[Temple University - Rome Campus]{walkintiber - 3D Digital Twin of Rome}{Aug 2024 -- Present}{Rome, ITA}
	\begin{itemize}
		\item Co-founded an immersive 3D model of Rome, allowing users to explore the city virtually via a web interface.
		\item Built a pipeline integrating live data from OpenStreetMap (OSM) to continuously update and enhance the model with realistic textures and digital elevation, supported by generative AI.
		\item Engaged TUR students and faculty in "walks in Rome" for hands-on learning in data mapping and VR technologies.
	\end{itemize}
	\smallskip
\fi

\bigskip

\cvsection{Education}

\iflongversion

	\cvevent[Sapienza University -- DIMA]{Ph.D.\ in Theoretical \& Applied Mechanics}{Nov 2017 -- May 2021}{Rome, ITA}
	Thesis title: ``Droplet condensation in turbulent jets.'' Supervised by: Prof. Paolo Gualtieri. \\
	\smallskip
	\begin{itemize}
		\item Developing high-performance software (HPC) for simulating complex, high-resolution, multi-phase turbulent flows.
		\item Maintaining optimized codebases, using parallel computing paradigms (OpenMP, MPI, CUDA) to take full advantage of distributed computing facilities
		\item Deriving an analytical approach to accurately represent phase transition processes interaction, in turbulent flows.
		\item Building a pipeline to deploy the compiled code onto remote supercomputers, running simulations on different environments and to periodically post-process intermediate results, monitoring the status real-time.
	\end{itemize}
	\medskip
	\cvtag{Turbulence}\cvtag{Multiphase Flows}\cvtag{HPC}\cvtag{Direct Numerical Simulations}\cvtag{Fortran}\
	\medskip

	\cvevent{M.Sc.\ in Aeronautical Engineering}{Sept 2014 -- July 2017}{Rome, ITA}
	\textbf{110/110 cum laude} Thesis title: ``A consistent framework for mass, momentum and energy exchange in two phase flows: DNS of a turbulent jet in the two-way coupling regime.'' Supervised by: Prof Carlo Massimo Casciola. \\
	\smallskip
	\begin{itemize}
		\item Specialized into aerodynamics, propulsion, and structures (APS) curriculum.
		\item Focused on continuum mechanics, applying fundamental science to solve complex engineering problems.
		\item Fascinated by theoretical approach and numerical methods.
	\end{itemize}
	\medskip
	\cvtag{Fortran}\cvtag{PDE}\cvtag{Navier-Stokes Eqs.}

	\divider

	\cvevent[ISAE Supaero -- DAEP]{Erasmus+ Program}{Jan 2016 -- Aug 2016}{Toulouse, FRA}
	Project title: ``Cost versus accuracy in LES of wall bounded flows in CharlesX.'' Supervised by: Prof. Julien Bodart. \\
	\medskip
	\begin{itemize}
		\item Studied in an international environment, in the heart of an aerospace hub in Europe, with many different cultures.
		\item Experienced with enthusiasms the multidisciplinary  approach, that is typical of France and that has enriched me so much from a technical point of view.
		\item Developed Python code for post-processing fluid dynamics numerical simulations, benchmarking a software developed by the Center for Turbulence Research (CTR), in Stanford.
	\end{itemize}
	\medskip
	\cvtag{python}\cvtag{C++}\cvtag{bash}\cvtag{French}

	\divider

	\cvevent[Sapienza University]{B.Sc.\ in Aerospace Engineering}{Sept 2011 -- Nov 2014}{Rome, ITA}
	\textbf{110/110 cum laude} Thesis title: ``The Newton's problem of minimal resistance.'' Supervised by: Prof. Andrea Dall'Aglio. \\
	\medskip
	% Aerospace Engineering is where the convergence of hard science and diverse technologies takes place to advance humanity's outer reach, with a focus on integrating diverse technological solutions in a collaborative, often international environment. This dual approach reflects my belief that true innovation often comes from a deep understanding of fundamental principles and culminated in my BSc thesis.
	% \medskip
	\cvtag{Functional Analysis}\cvtag{\LaTeX}

	\divider

	\cvevent[LSS J.F. Kennedy]{High School Diploma}{Sept 2006 -- July 2011}{Rome, ITA}
	\textbf{87/100} Baccalaureate thesis: ``Revolutions in history and scientific thought.''
	\medskip\
	\begin{itemize}
		\item Playing competitive water polo with daily training. I learned a lot about cooperation, discipline and resilience, both in and out of the pool.
		\item Advanced education program (PNI), including intensive study of scientific subjects such as mathematics and physics, complemented by the use of computer science as a tool.
	\end{itemize}

	\bigskip

	\columnratio{0.55}
	% Start a 2-column paracol. Both the left and right columns will automatically break across pages if things get too long.
	\begin{paracol}{2}
		% Two-column content goes here

		\cvsection{Referees}

		% \cvref{name}{email}{mailing address}
		\cvref{Prof.\ Andrea Dall'Aglio}{Sapienza University}{dallaglio@mat.uniroma1.it}
		%{Address Line 1\\Address line 2}
		\cvref{Prof.\ Mary Conran}{Temple University Rome Campus}{mary.conran@temple.edu}
		%{Address Line 1\\Address line 2}
		\cvref{Prof.\ Carlo Massimo Casciola}{Sapienza University}{carlomassimo.casciola@uniroma1.it}
		\cvref{Prof.\ Paolo Gualtieri}{Sapienza University}{paolo.gualtieri@uniroma1.it}
		\cvref{Leonardo Moavero}{IDS AirNav}{leonardo.moavero@idsairnav.com}
		\cvref{Valerio Ferrara}{ENAV Group}{valerio.ferrara@enav.it}
		\cvref{Prof.\ Julien Bodart}{ISAE Supaero}{julien.bodart@isae.fr}

		\switchcolumn

		\cvsection{Languages}

		\cvskill{English (Full Professional)}{4.5}
		\cvskill{French}{3.5} %% Supports X.5 values.
		\cvskill{Italian (Native)}{5}

		% \cvsection{Strengths}

		% \cvtag{Hard-working}
		% \cvtag{Eye for detail}\\
		% \cvtag{Motivator \& Leader}

		% \divider\smallskip

		% \cvtag{C++}
		% \cvtag{Embedded Systems}\\
		% \cvtag{Statistical Analysis}

		\bigskip

		\cvsection{More}

		I love working in diverse teams, fostering growth, and building strong relationships with my colleagues. \par

		\smallskip

		Outside of work, you'll find me cooking with friends and family, discussing movies of all genres, staying active, or diving into late night coding sessions. \par

		\smallskip

		I'm always looking for opportunities to grow, share knowledge, and make a positive impact. I free-dive into the blue and look around. I like to take action.

	\end{paracol}

\else

	\cvevent[Sapienza University -- DIMA]{Ph.D.\ in Theoretical \& Applied Mechanics}{Nov 2017 -- May 2021}{Rome, ITA}
	``Droplet condensation in turbulent jets.'' \\
	\medskip
	\cvtag{HPC}\cvtag{MPI}\cvtag{C}\cvtag{Multiphase Flows}

	\divider

	\cvevent[Sapienza University -- DIMA]{M.Sc.\ in Aeronautical Engineering}{Sept 2014 -- July 2017}{Rome, ITA}
	``A consistent framework for mass, momentum and energy exchange in two phase flows: DNS of a turbulent jet in the two-way coupling regime.'' \\
	\medskip
	110/110 cum laude \\
	\medskip
	\cvtag{Fortran}\cvtag{PDE}\cvtag{Navier-Stokes Eqs.}

	\divider

	\cvevent[ISAE Supaero -- DAEP]{Erasmus+ Program}{Jan 2016 -- Aug 2016}{Toulouse, FRA}
	``Cost versus accuracy in LES of wall bounded flows in CharlesX.'' \\
	\medskip
	\cvtag{python}\cvtag{C++}\cvtag{bash}\cvtag{French}

	\divider

	% \cvevent[Sapienza University]{B.Sc.\ in Aerospace Engineering}{Sept 2011 -- Nov 2014}{Rome, ITA}
	% ``The Newton's problem of minimal resistance.'' \\
	% \medskip
	% 110/110 cum laude \\
	% \medskip
	% \cvtag{Functional Analysis}\cvtag{\LaTeX}

	% \divider

	% \cvevent[LSS J.F. Kennedy]{High School Diploma}{Sept 2006 -- July 2011}{Rome, ITA}
	% 87/100

	\bigskip

	\cvsection{Languages}

	\cvskill{English (Full Professional)}{4.5}
	\cvskill{French}{3.5} %% Supports X.5 values.
	\cvskill{Italian (Native)}{5}

\fi

\divider

Last updated: \today

% \cvsection{My Life Philosophy}

% \begin{quote}
% 	``Whatever Works.'' comedy film directed and written by Woody Allen
% \end{quote}

% \cvsection{Most Proud of}

% \cvachievement{\faTrophy}{Fantastic Achievement}{and some details about it}

% \divider

% \cvachievement{\faHeartbeat}{Another achievement}{more details about it of course}

% \divider

% \cvachievement{\faHeartbeat}{Another achievement}{more details about it of course}

% %% Yeah I didn't spend too much time making all the
% %% spacing consistent... sorry. Use \smallskip, \medskip,
% %% \bigskip, \vspace etc to make adjustments.
% \medskip

	\end{paracol}
\fi

\end{document}
